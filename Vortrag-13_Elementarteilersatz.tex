\documentclass[11pt,a4paper,ngerman,DIV=11]{scrreprt}
\usepackage[utf8]{inputenc}
\usepackage[T1]{fontenc}
\usepackage[ngerman]{babel}
\usepackage{amsmath}
\usepackage{amssymb}
\usepackage{amsthm}
\usepackage{mathtools}
\usepackage{array}
\usepackage[numbers,sort&compress]{natbib}
\usepackage[babel]{csquotes}
%\usepackage{enumerate} % TODO: replace with better 'enumitem'
\usepackage[shortlabels]{enumitem}
\usepackage{ifmtarg}
%\usepackage[pdftex]{graphicx}
\usepackage[all]{xy}
\usepackage{gauss}

\usepackage[pdftex,bookmarks,colorlinks=false,pdfborder={0 0 0},%
            pdftitle={Proseminar Modultheorie - Vortrag 13: Elementarteilersatz},%
            pdfauthor={Johannes Prem}]{hyperref}
%
\usepackage{cleveref}

\newcommand{\pref}[1]{(\ref{#1})}
\newcommand{\pcref}[1]{(\cref{#1})}

\setlength{\parindent}{0pt}
\setlength{\parskip}{0.5em}

\newcommand{\origepsilon}{} % keine LaTeX-Kollisionen
\let\origepsilon=\epsilon
\let\epsilon=\varepsilon

\let\origphi=\phi
\let\phi=\varphi

\renewcommand{\qedsymbol}{$\blacksquare$}

\newtheoremstyle{mythms}
 {15pt}% Space above
 {12pt}% Space below 
 {}% Body font
 {}% Indent amount: Indent amount: empty = no indent, \parindent = normal paragraph indent
 {\bfseries}% Theorem head font
 {.}% Punctuation after theorem head
 {0.6cm}% Space after theorem head: Space after theorem head: { } = normal interword space; \newline = linebreak
 {}% Theorem head spec (can be left empty, meaning `normal')
 
\theoremstyle{mythms}
\newtheorem{globalnum}{DUMMY DUMMY DUMMY}[chapter]
\newtheorem{thDef}[globalnum]{Definition}
\newtheorem{thSatz}[globalnum]{Satz}
%\newtheorem{thPropos}[globalnum]{Proposition}
\newtheorem{thLemma}[globalnum]{Lemma}
\newtheorem{thKorollar}[globalnum]{Korollar}

\newtheorem{thBemerkung}[globalnum]{Bemerkung}
\newtheorem{thBeisp}[globalnum]{Beispiel}
\newtheorem{thBeispiele}[globalnum]{Beispiele}
\newenvironment{BspList}{%
\nopagebreak\begin{thBeispiele}%
\hfill\begin{enumerate}[a),parsep=0pt,itemsep=0.8ex,leftmargin=2em]%
}{%
\end{enumerate}\end{thBeispiele}
}
%

\newenvironment{proofsketch}[1][]{%
\begin{proof}[Beweisskizze#1]
}{%
\end{proof}
}

\makeatletter
\let\origthmhead=\thmhead
\renewcommand{\thmhead}[3]{%
\phantomsection
\origthmhead{#1}{#2}{#3}%
\pdfbookmark[1]{#1\@ifnotempty{#1}{ }#2\thmnote{ (#3)}}{#1#2}%
}
\makeatother

\DeclareMathOperator*{\Exists}{\exists}
\DeclareMathOperator*{\forAll}{\forall}

\newcommand{\Mid}{\,\middle\vert\,}
%\newcommand{\Mid}{\mathrel{\middle\vert}}

%\makeatletter
%\newcommand{\defmathbbsymbol}[1]{%
%    \expandafter\newcommand\csname #1\endcsname[1][]{%
%        \@ifmtarg{##1}{\mathbb{#1}}{\mathbb{#1}_{##1}}%
%    }%
%}
%\makeatother
\newcommand{\defmathbbsymbol}[1]{%
    \expandafter\newcommand\csname #1\endcsname{\mathbb{#1}}%
}


\defmathbbsymbol{N}
\defmathbbsymbol{Z}
\defmathbbsymbol{Q}
\defmathbbsymbol{R}
\defmathbbsymbol{C}

\newcommand{\mr}{\mathrm}

\DeclarePairedDelimiter{\abs}{\lvert}{\rvert}
\DeclarePairedDelimiter{\Spann}{\langle}{\rangle}

\DeclareMathOperator{\Kern}{ker}
\DeclareMathOperator{\Image}{im}

\DeclareMathOperator{\modulelength}{\ell}

\newcommand{\barfrak}[1]{\bar{\mathfrak{#1}}}

\newcommand{\longto}{\longrightarrow}

\newcommand{\len}[1][R]{\modulelength_{#1}}

\newcommand{\longhookrightarrow}{\lhook\joinrel\relbar\joinrel\rightarrow}

\newcommand{\ZRest}[1]{\Z/#1\Z}

\newcommand*{\Matrix}[1]{%
\begin{pmatrix} #1 \end{pmatrix}%
}
\newcommand{\vect}{}
\let\vect=\Matrix

\newcommand{\D}{\mr{d}}
\newcommand{\A}[2]{a_{#1#2}}

\newcommand{\downsquigarrow}{\rotatebox{-90}{$\rightsquigarrow$}}

% make gmatrix right aligned
\makeatletter

\let\g@post@orig=\g@post

\newcommand{\gmatrixright}{%
    \global\edef\g@post{\relax$}
}
\newcommand{\gmatrixcenter}{%
    \global\let\g@post=\g@post@orig
}

\makeatother


%%%%%

\makeatletter

\newcommand{\matops@colstart}{\begin{gmatrix}}
\newcommand{\matops@colend}{\end{gmatrix}&}
\newcommand{\matops@newline}{\end{gmatrix}\\\matops@turnsinto}
\newcommand{\matops@end}{\end{gmatrix}}

\newlength{\arrowpos}
\setlength{\arrowpos}{5cm}

\newcommand{\matops@turnsinto}{%
\noalign{\vskip7pt\hbox to \arrowpos{\hfill\downsquigarrow}\vskip7pt}%
}

\newcommand{\nextcol}{\matops@colend}
\newcommand{\nextstep}{\matops@newline}
\newcommand{\continued}{\matops@turnsinto}

\newsavebox{\matops@matrixh}
\newlength{\matops@matrixheight}
\savebox{\matops@matrixh}{$\displaystyle\begin{gmatrix}1\\2\\3\end{gmatrix}$}
\setlength{\matops@matrixheight}{\ht\matops@matrixh}
\addtolength{\matops@matrixheight}{\dp\matops@matrixh}

\newcommand{\matops@interrule}{%
\rule[-0.8\dp\matops@matrixh]{0.4pt}{0.9\matops@matrixheight}}

\newcommand{\matops@interspaceleft}{\quad}
\newcommand{\matops@interspaceright}{\qquad}

\newcommand{\matops@leftparen}{%
\ensuremath{\displaystyle\left(\vphantom{\matops@interrule}\right.}}
\newcommand{\matops@rightparen}{%
\ensuremath{\displaystyle\left.\vphantom{\matops@interrule}\right)}}
    
\newcolumntype{R}{>{\matops@colstart}r}
\newcolumntype{C}{>{\matops@colstart}c}
\newcolumntype{L}{>{\matops@colstart}l}

\newenvironment{trimatrixoperations}[1][L]{%
\renewcommand{\|}{\matops@colend}
\begin{array}{@{\matops@leftparen\,}%
              R%
              @{\matops@interspaceleft\matops@interrule\matops@interspaceright}%
              #1%
              @{\matops@interspaceleft\matops@interrule\matops@interspaceright}%
              L%
              @{\kern-4pt\matops@rightparen}}
}{%
\matops@end
\end{array}
}

\makeatother

\newcommand{\vzfix}{\phantom{-}}

%\newcommand{\p}[1]{\cdot(#1)}
%\newcommand{\m}[1]{\cdot(-#1)}

%%%%%


\setlist[itemize,1]{label=--}

\SelectTips{cm}{}
\UseTips
\setcounter{secnumdepth}{0}

\begin{document}


\chapter{Vorwort und Notation}
Dieses Skript soll den Inhalt von Bosch\cite[S.\,206\,ff., 6.3]{bookc:bosch08}
zusammenfassen. Deshalb werden mehrfach Beweise nicht im Detail ausgeführt und
dabei auf \cite{bookc:bosch08} verwiesen.

Weiter sei angemerkt, dass der Beweis des Elementarteilersatzes in
Bosch\cite{bookc:bosch08} sehr \enquote{intuitiv} ist, so dass er zwar leicht zu
verstehen ist, mehrfach jedoch eher mathematisch unvollständig erscheint.
Wer gerne einen konzeptionelleren Beweis lesen möchte, kann dies entweder in
\cite[Kap.~7]{lecnotes:gub:alg2} oder in \cite[Kap.~22]{lecnotes:jan:la2} tun.
In der letzten Referenz wird insbesondere auf den Seiten~102--121 alles Nötige
entwickelt, um den dann folgenden Beweis auch ohne von der der Linearen Algebra~II
abweichendes Vorwissen verstehen zu können.

\bigskip
In diesem Skript wird folgende Notation verwendet:
\begin{itemize}
    \item
        Sowohl $\subset$ als auch $\subseteq$ stehen für: enthalten oder gleich.
        Echt enthalten wird durch $\subsetneq$ gekennzeichnet.

    \item
        Die \emph{Natürlichen Zahlen $\N$} beginnen mit $0$.

    \item
        Äquivalenzklassen werden durch Überstriche gekennzeichnet, zum Beispiel:
        $\bar 1, \bar c \in \ZRest3$.

    \item
        Zu einem Ring $R$ bezeichnet $R^\times$ die Einheitengruppe des Rings.
\end{itemize}


\chapter{Vorbereitungen}
Es sei $R$ im Folgenden immer ein kommutativer Ring. Damit ist auch klar, dass
wir nicht zwischen $R$-Links- und $R$-Rechtsmoduln unterscheiden müssen, so dass
wir einfach nur $R$-Modul schreiben werden.

\begin{thDef}[Länge eines Moduls]
    Sei $M$ ein $R$-Modul. Definiere nun $\len(M)$ als das Supremum der Längen
    $\ell$ von echt aufsteigenden Ketten von Untermoduln der Form:
    \[ 0 \subsetneq M_1 \subsetneq M_2 \subsetneq \cdots \subsetneq M_\ell = M \]
    Dabei lassen wir auch $\infty$ als Wert zu, so dass also $\len(M) \in
    \N\cup\{\infty\}$.
\end{thDef}

\begin{thBemerkung}
    Man kann die \emph{Länge eines Moduls} auch über nicht kommutativen Ringen
    einführen. Dies wird im Folgenden aber nicht benötigt.
\end{thBemerkung}

\begin{BspList}
\item
    Es ist offenbar $\len(M) = 0 \iff M = 0$.
\item
    Für den Sonderfall eines $n$-dim. $K$-Vektorraums $V$ (aufgefasst als
    $K$-Modul) hat man $\len[K](V) = n$, wie man schnell mit Hilfe des bekannen
    Isomorphismus $V \cong K^n$ und dessen Zerlegung $0\subsetneq K \subsetneq
    K^2 \subsetneq \cdots \subsetneq K^n$ sieht.
\item
    Für $\Z$ als $\Z$-Modul hat man die Länge $\infty$ und für $\ZRest2$ als
    $\ZRest2$-Modul erhält man die Länge $1$, da $\ZRest2$ keine echten
    Untermoduln besitzt.
\end{BspList}

Wir werden die Länge als Hilfsmittel brauchen, um den Elementarteilersatz
% TODO: \ref einfügen
zu zeigen. Dazu benötigen wir noch zwei einfache Lemma über die Länge:

\begin{thLemma}%[Länge des Restklassenmoduls über Hauptidealringen]
    Sei $R$ ein Hauptidealring\footnote{künftig abgekürzt durch: HIR} und $a\in
    R$ mit (bis auf Assoziiertheit eindeutiger) Primfaktorzerlegung 
    $a = p_1\cdots p_r$.
    % TODO: (Ex. nach \cite ..)
    Dann hat der Restklassenmodul $R/aR$ (über $R$) die Länge $\len(R/aR) = r$.
\end{thLemma}
\begin{proof}
    Zunächst ist klar, dass die Ideale in $R$ gerade den Untermoduln von $R$
    entsprechen, wenn man $R$ als Modul über sich selbst auffasst.

    Sei 
    \begin{align*}
        \pi\colon R &\to R/aR   \\
                  x &\mapsto \bar x
    \end{align*}
    der kanonische Epimorphismus. Dann kann man die Ideale $\barfrak{a} \subset
    R/aR$ mit Hilfe von $\pi$ unter der Bildung von $\pi^{-1}(\barfrak{a})$
    bijektiv den Idealen in $R$ zuordnen, die $aR$ enthalten. (nachrechnen!)
    
    Damit ist klar, dass man statt Idealketten des Typs
    \[ 0 \subsetneq \barfrak{a}_1 \subsetneq \barfrak{a}_2 \subsetneq 
        \cdots \subsetneq \barfrak{a}_\ell = R/aR      \]
    auch die folgenden Idealketten betrachten kann:
    \[ aR =: \mathfrak{a}_0 \subsetneq \mathfrak{a}_1 
                \subsetneq \mathfrak{a}_2 \subsetneq 
                \cdots \subsetneq \mathfrak{a}_\ell = R        \]
    Es genügt also, das Supremum der Längen derartiger Ketten zu betrachten. Da
    $R$ ein HIR ist, wird jedes dieser $\mathfrak{a}_i$ von einem Element $a_i$
    erzeugt: $\mathfrak{a}_i = \Spann{a_i}$. Da für $i\in\{0,\ldots \ell-1\}$ und
    die Ideale $\mathfrak{a}_i,\mathfrak{a}_{i+1}$ die Äquivalenz
    % TODO: \cite
    \[ \mathfrak{a}_i \subset \mathfrak{a}_{i+1} \iff a_{i+1} \mid a_i \]
    gilt und wir aber echte Inklusion betrachten, erhalten wir, dass auch
    $a_{i+1}$ ein echter Teiler von $a_i$ sein muss. Mit der vorausgesetzten
    Primfaktorzerlegung von $a$ sehen wir nun also leicht ein, dass wir gerade
    eine aufsteigende Kette von Idealen (die $aR$ enthalten) der Länge $r$
    bilden können.
    \\
\end{proof}

\begin{thLemma}[Additivität der Länge bei direkten Summen]
    Sei ein $R$-Modul $M$ die direkte Summe zweier Untermoduln $M'$ und $M''$.
    Dann gilt:
    \[ \len(M) = \len(M') + \len(M'') \]
\end{thLemma}
\begin{proof}
    Wir zeigen die Gleichheit in zwei Schritten:
    \begin{description}
        \item[\enquote{\boldmath$\geq$}:]
            Hat man für die Untermoduln aufsteigende Ketten wie folgt:
            \begin{align*}
                0 \subsetneq M_1^{\prime\phantom\prime} \subsetneq
                            M_2^{\prime\phantom\prime} 
                            \subsetneq \cdots 
                            \subsetneq M_r^{\prime\phantom\prime}  &= M'    \\
                0 \subsetneq M_1'' \subsetneq M_2'' 
                            \subsetneq \cdots \subsetneq M_s'' &= M''
                ,
            \end{align*}
            so kann man direkt eine aufsteigende Kette der Länge $r+s$ für $M$
            angeben:
            \[ 0 \subsetneq M_1' \oplus 0 \subsetneq \cdots M_r' \oplus 0
                \subsetneq M_r' \oplus M_1'' \subsetneq M_r' \oplus M_s'' = M \]
            Es folgt: $\len(M) \geq \len(M') + \len(M'')$
            
        \item[\enquote{\boldmath$\leq$}:]
            Um diese Relation zu zeigen, betrachten wir eine aufsteigende Kette
            von Untermoduln von $M$:
            \[ 0 =: M_0 \subsetneq M_1 \subsetneq M_2 \subsetneq \cdots
            \subsetneq M_\ell = M \]
            Dann bekommen wir zu jedem $i\in\{0,\ldots,\ell-1\}$ ein anschauliches
            Diagramm wie folgt:
            \begin{equation*}
                \begin{xy}
                    \xymatrix@C=5pt@R=13pt{
                        M_i\cap M' \ar@{}[d]|-*[@][*1.2]{\subseteq}  & \subset
                        & M_i      \ar@{}[d]|-*[@][*1.2]{\subsetneq} & \overset\pi\longto
                        & \pi(M_i) \ar@{}[d]|-*[@][*1.2]{\subseteq}
                        \\
                        M_{i+1}\cap M' & \subset
                        & M_{i+1} & \overset\pi\longto
                        & \pi(M_{i+1})
                    }
                \end{xy}
            \end{equation*}
            Dabei bezeichne $\pi\colon M'\oplus M'' \to M''$ die Projektion auf
            $M''$. Wir bemerken dabei schon einmal, dass dann gerade 
            $\Kern(\pi) = M'$ ist.
            
            Nehmen wir nun einmal an, dass sowohl $M_i\cap M' = M_{i+1}\cap M'$
            als auch $\pi(M_i) = \pi(M_{i+1})$ in obigem Diagramm gilt.
            Wir zeigen nun, dass dies schon $M_i = M_{i+1}$ zur Folge hätte, was
            aber nach Voraussetzung ausgeschlossen ist.
            
            Sei also $x\in M_{i+1}$. Dann haben wir wegen der Gleichheit
            $\pi(M_i) = \pi(M_{i+1})$ ein $x'\in M_i$, so dass $\pi(x) =
            \pi(x')$ gilt. Da $\pi$ linear ist, also auch: $\pi(x-x') = 0 \iff
            x-x' \in \Kern(\pi) = M'$. Daraus folgt nun, dass $x$ und $x'$ auch
            in $M_{i+1}\cap M'$ sein müssen und wegen der Gleichheit 
            $M_i\cap M' = M_{i+1}\cap M'$ also auch in $M_i \cap M'$. Dann ist
            aber auch $x-x' \in M_i\cap M' \subset M_i$. Wenn aber $x'$ und
            $(x-x')$ in $M_i$ liegen, so auch $x = x' + (x-x') \in M_i$.
            Damit wäre $M_{i+1} \subset M_i$. Widerspruch.
            
            Also muss die Annahme falsch gewesen sein und es folgt:
            \[ \forAll_{i\in\{0,\ldots,\ell-1\}} \;
                M_i\cap M' \subsetneq M_{i+1}\cap M'  \;\vee\;
                \pi(M_i)   \subsetneq \pi(M_{i+1})  \]
            Daraus ergibt sich wie gewünscht:
            \[ \ell \leq \len(M') + \len(M'') \]
    \end{description}
    Insgesamt haben wir also Gleichheit, d.\,h. die Behauptung.
    \\
\end{proof}

\begin{thBemerkung}
    Kennt man das \emph{Fünferlemma}, so kann man den zweiten Schritt im
    vorangegangenen Beweis deutlich verkürzen, indem man einfach das folgende
    kommutative Diagramm mit exakten Zeilen betrachtet ($\pi$ wie oben):
    \begin{equation*}
        \newcommand{\xyhookarrowdown}{\ar@{^{(}->}[]-<0pt,11pt>;[d]+<0pt,11pt>}
        \begin{xy}
            \xymatrix@C=5pt@R=20pt{
                0                                   & \longto
                & M_i\cap M'    \xyhookarrowdown    & \longhookrightarrow
                & M_i           \xyhookarrowdown    & \overset\pi\longto
                & \pi(M_i)      \xyhookarrowdown    & \longto
                & 0
                \\
                0                                   & \longto
                & M_{i+1}\cap M'                    & \longhookrightarrow
                & M_{i+1}                           & \overset\pi\longto
                & \pi(M_{i+1})                      & \longto
                & 0
            }
        \end{xy}
    \end{equation*}
\end{thBemerkung}

Wir können nun zum eigentlichen Thema kommen:








\chapter{Der Elementarteilersatz}
\begin{thSatz}[Elementarteilersatz]
    Sei $F$ ein endlich erzeugter freier Modul über einem HIR $R$. Sei außerdem
    $M\subset F$ ein Untermodul. Dann existieren Elemente $x_1,\ldots,x_s\in F$
    und $\alpha_1,\ldots,\alpha_s\in R\setminus\{0\}$ mit folgenden
    Eigenschaften:
    \begin{enumerate}[i)]
        \item
            $x_1,\ldots,x_s$ sind Teil einer Basis von $F$.
        \item
            $\alpha_1 x_1, \ldots, \alpha_s x_s$ bilden eine Basis von $M$.
        \item
            Es gilt für alle $i\in\{1,\ldots,s-1\}$:\; $\alpha_i\mid\alpha_{i+1}$.
    \end{enumerate}
    Außerdem sind die $\alpha_1,\ldots,\alpha_s$ bis auf Assoziiertheit
    eindeutig durch $M$ bestimmt (unabhängig von der Wahl der $x_1,\ldots,x_s$)
    und man nennt diese Elemente dann die \emph{Elementarteiler} von 
    $M\subset F$.
\end{thSatz}

Der Beweis dieses Satzes wird für die Existenz- und Eindeutigkeitsaussage
getrennt geführt und erfordert einigen Aufwand. Es soll hier deshalb nur eine
Beweisskizze gegeben werden, in der die wichtigen Schritte dargestellt werden.
Einen kompletten Beweis nach dem folgenden Schema kann man in 
% TODO: \cite
[TODO]
nachlesen.

\begin{proofsketch}[ zur Existenzaussage]
    Wir wählen zunächst eine Basis $Y = (y_1,\ldots,y_m)$ von~$F$. Per Induktion
    über $m$ zeigt man nun, dass auch jeder Untermodul $M\subset F$ endlich
    erzeugt ist: für $m=0$ ist nichts zu zeigen und für $m=1$ ist die Behauptung
    klar, da dann $F\cong R$ ist und somit gilt, dass $M$ gerade einem Ideal in
    $R$ entspricht, welches aber nach Voraussetzung von einem einzigen Element erzeugt
    wird; Für $m>1$ betrachtet man dann eine geeignete Zerlegung $F = F'
    \oplus F''$, wendet auf $M\cap F'$ und die Projektion von $M$ auf $F''$ die
    Induktionsvoraussetzung an und zeigt dann, dass man als Vereinigung von
    endlichen Erzeugendensystemen dieser beiden Untermoduln ein
    Erzeugendensystem für ganz $M$ erhält.
    
    Da wir nun wissen, dass $M$ endlich erzeugt ist, können wir ein solches
    Erzeugendensystem $z_1,\ldots,z_n$ wählen. Weiter bezeichne $e_1,\ldots,e_n$
    die Standardbasis von $R^n$ als $R$-Modul, so kann man eine lineare
    Abbildung $f$ wie folgt definieren:
    \begin{align*}
        f\colon R^n &\to F  \\
        e_j &\mapsto z_j
    \end{align*}
    <++>
    \\
\end{proofsketch}














\chapter{Beispiele und Anwendungen}
Ein erstes Beispiel haben wir schon in [TODO] % TODO: \cite 08
gesehen. Dort wurden durch einfache Umformungen die Elementarteiler einer Matrix
bestimmt:

\begin{thBeisp}[siehe auch {[TODO]}] % TODO: \cite 08/2.6
    Sei 
    \[ A := \Matrix{
           6 & 2 &  5 \\
          32 & 2 & 28 \\
          30 & 2 & 26  } \in \Z^{3\times3}
    \,. \]
    Mit dem beschriebenen Algorithmus erhält man dann:
    \[
        A
        \rightsquigarrow \ldots
        \rightsquigarrow 
    \Matrix{
        1 & 0 & 0 \\
        0 & 2 & 0 \\
        0 & 0 & 6  }
    \]
    Also sind die Elementarteiler: $1,2,6 \in\Z$.
\end{thBeisp}

Wir wollen uns daher gleich einem weiteren Beispiel zuwenden, in welchem
wir dann den Algorithmus aus dem Beweis zu Lemma % TODO: \ref
nochmal explizit durchführen.

\begin{thBeisp}
    Betrachte den $\Z$-Modul $\Z^3$. Seien $v_1,v_2,v_3 \in\Z^3$ und der
    Untermodul $M\subset\Z^3$ wie folgt definiert:
    \begin{gather*}
        v_1 := \vect{2\\-2\\0}, \quad 
        v_2 := \vect{0\\ 0\\7}, \quad
        v_3 := \vect{8\\-2\\1},
        \\[2mm]
        M := \Spann[\big]{v_1,v_2,v_3}_{\Z}
    \end{gather*}
    Wir möchten nun die in Satz % TODO: \ref
    behauptete Basis $x_1,\ldots,x_3$ von $\Z^3$ und Elemente
    $\alpha_1,\alpha_2,\alpha_3\in\Z$ bestimmen, so dass 
    $\alpha_1 x_1,\alpha_2 x_2,\alpha_3 x_3$ eine Basis von $M$ bilden.
    
    Dazu betrachten wir die lineare Abbildung, dargestellt durch die Matrix
    \[ A := \Matrix{ 2 & 0 &  8 \\
                    -2 & 0 & -2 \\
                     0 & 7 &  1  }  , \]
    welche offensichtlich die Standardbasis von $\Z^3$ auf das Erzeugendensystem
    $v_1,v_2,v_3$ von $M$ abbildet. Wir können nun das oben beschriebene
    Verfahren anwenden, um $A'$ zu erhalten. Diesmal sind wir aber zusätzlich an
    den Transformationsmatrizen $S$ und $T$ interessiert, so dass $S\!AT = A'$
    gilt. (Wieso wird weiter unten klar werden.) Um $S$ und $T$ gleich
    zusätzlich zu $A'$ zu erhalten, setzen wir $S_0:=T_0:=E_3$ (wobei $E_3$ die
    $(3\times3)$-Einheitsmatrix bezeichnet) und betrachten das erweiterte System
    $(S_0|A|T_0)$. Wir wenden dann alle Operationen, die Zeilen (Spalten)
    betreffen, zusätzlich auf $S_0$ ($T_0$) an bekommen so die gewünschten 
    Transformationsmatrizen gleich mit dazu.
    %
    (Dass dies funktioniert, wird ersichtlich, wenn man sich noch einmal klarmacht,
    dass eigentlich alle durchgeführten Operationen einer Matrixmultiplikation
    entsprechen: von links bei Zeilen- und von recht bei Spaltenumformungen.)
    
    Es wird nun der Algorithmus für $A$ durchlaufen. Dabei stehen die nötigen
    Umformungen zwar immer nur bei der Matrix $A$, sie werden aber natürlich
    entsprechend auch auf $S_0$ und $T_0$ angewendet. Es ergibt sich also:
    
    \begin{equation*}
        \begin{trimatrixoperations}
             1 &  0 &  0 \\ 
             0 &  1 &  0 \\
             0 &  0 &  1
            \|
             2 &  0 &  8 \\
            -2 &  0 & -2 \\
             0 &  7 &  1
             \rowops
             \swap{0}{2}
             \colops
             \swap{0}{2}
            \|
             1 &  0 &  0 \\ 
             0 &  1 &  0 \\
             0 &  0 &  1
            %%%
            \nextstep
            %%%
             0 &  0 &  1 \\
             0 &  1 &  0 \\
             1 &  0 &  0  
            \|
             1 &  7 &  0 \\
            -2 &  0 & -2 \\
             8 &  0 &  2
             \colops
             \add[-7]{0}{1}
            \|
             0 &  0 &  1 \\
             0 &  1 &  0 \\
             1 &  0 &  0  
            %%%
            \nextstep
            %%%
             0 &  0 &  1 \\
             0 &  1 &  0 \\
             1 &  0 &  0  
            \|
             1 &   0 &  0 \\
            -2 &  14 & -2 \\
             8 & -56 &  2
             \rowops
             \add[2]{0}{1}
             \add[-8]{0}{2}
            \|
             0 &  0 &  1 \\
             0 &  1 &  0 \\
             1 & -7 &  0  
            %%%
            \nextstep
            %%%
             0 &  0 &  1 \\
             0 &  1 &  2 \\
             1 &  0 & -8  
            \|
            \vzfix
             1 &   0 &  0 \\
             0 &  14 & -2 \\
             0 & -56 &  2
             \colops
             \swap{1}{2}
            \|
             0 &  0 &  1 \\
             0 &  1 &  0 \\
             1 & -7 &  0  
            %%%
            \nextstep
            %%%
             0 &  0 &  1 \\
             0 &  1 &  2 \\
             1 &  0 & -8  
            \|
            \vzfix
             1 &  0 &   0 \\
             0 & -2 &  14 \\
             0 &  2 & -56
             \colops
             \add[7]{1}{2}
            \|
             0 &  1 &  0 \\
             0 &  0 &  1 \\
             1 &  0 & -7  
            %%%
            \nextstep
            %%%
             0 &  0 &  1 \\
             0 &  1 &  2 \\
             1 &  0 & -8  
            \|
            \vzfix
             1 &  0 &   0 \\
             0 & -2 &   0 \\
             0 &  2 & -42
             \rowops
             \add{1}{2}
            \|
             0 &  1 &  7 \\
             0 &  0 &  1 \\
             1 &  0 & -7  
        \end{trimatrixoperations}
    \end{equation*}
    \begin{equation*}
        \begin{trimatrixoperations}
            %%%
            \continued
            %%%
             0 &  0 &  1 \\
             0 &  1 &  2 \\
             1 &  1 & -6  
            \|
            \vzfix
             1 &  0 &   0 \\
             0 & -2 &   0 \\
             0 &  0 & -42
             \rowops
             \mult{1}{\cdot(-1)}
             \colops
             \mult{2}{\cdot(-1)}
            \|
             0 &  1 &  7 \\
             0 &  0 &  1 \\
             1 &  0 & -7  
            %%%
            \nextstep
            %%%
             0 &  0 &  1 \\
             0 & -1 & -2 \\
             1 &  1 & -6  
            \|
            \vzfix
             1 &  0 &   0 \\
             0 &  2 &   0 \\
             0 &  0 &  42
            \|
             0 &  1 & -7 \\
             0 &  0 & -1 \\
             1 &  0 &  7  
        \end{trimatrixoperations}
    \end{equation*}
    
    \medskip
    Wir erhalten also die Elementarteiler $\alpha_1 = 1,\; \alpha_2 = 2$ und
    $\alpha_3 = 42$ und zusätzlich die Transformationsmatrizen
    \[ S = \Matrix{
             0 &  0 &  1 \\
             0 & -1 & -2 \\
             1 &  1 & -6  } 
       \quad\text{und}\quad
       T = \Matrix{
             0 &  1 & -7 \\
             0 &  0 & -1 \\
             1 &  0 &  7  }
    . \]
    (Streng genommen haben wir im letzten Schritt der obigen Umformungen
    außerhalb des beschriebenen Verfahrens operiert; dass die Multiplikation
    einer Zeile oder Spalte mit einer Einheit auch erlaubt ist, sieht man aber
    schnell ein.)
    
    Nun ist aber $S$ gerade die Basiswechselmatrix von der neuen Basis
    $x_1,x_2,x_3$ in die Standardbasis und wir müssen somit nur noch $S$
    invertieren und können dann die passenden Elemente $x_1,x_2,x_3$ in den Spalten
    von $S^{-1}$ ablesen. (Es bleibt dem Leser überlassen, ein
    Invertierungsverfahren konkret anzuwenden; es bietet sich beispielsweise
    [Loher] % TODO: \cite 10/2.8
    an.) Man bekommt dann:
    \[ S^{-1} = \Matrix{
                     8 &  1 & 1 \\
                    -2 & -1 & 0 \\
                     1 &  0 & 0  }  \]
    Da nun $S^{-1}$ gerade von der Standardbasis in die neue Basis wechselt,
    haben wir für letztere:
    \[
        x_1 := \vect{8\\-2\\1}, \quad 
        x_2 := \vect{1\\-1\\0}, \quad
        x_3 := \vect{1\\ 0\\0}    \]
    Da wir die Elementarteiler schon haben, können wir nun auch sofort die
    gewünscht Basis für $M$ angeben:
    \[
        \alpha_1 x_1 := \vect{8\\-2\\1}, \quad 
        \alpha_2 x_2 := \vect{2\\-2\\0}, \quad
        \alpha_3 x_3 := \vect{42\\0\\0}    \]

    Wir haben nun zwar $T$ gar nicht benötigt, jedoch wird im nächsten Beispiel 
    klar werden, warum wir uns die Mühe gemacht haben, auch $T$ zu berechnen.
\end{thBeisp}

Kommen wir nun zu einer sehr anschaulichen Anwendung des Elementarteilersatzes.
Man kann diesen nämlich auch dazu verwenden, sog. \emph{lineare Diophantische
Gleichungen} konstruktiv zu lösen. Darunter versteht man lineare
Gleichungssysteme mit Koeffizienten aus~$\Z$, wobei man auch nur Lösungen aus
$\Z$ sucht. Konkret:

\begin{thBeisp}
    Wir suchen ganzzahlige Lösungen des folgenden Gleichungssystems:
    \begin{alignat*}{4}
         &2x_1  &\quad&      &\quad&            +8x_3  &\quad=&\quad \; 62             \\
        -&2x_1  &\quad&      &\quad&            -2x_3  &\quad=&\quad            {-2}   \\
         &      &\quad& 7x_2 &\quad& +\phantom{1} x_3  &\quad=&\quad \phantom{+}  3
    \end{alignat*}
    %
    Wir können dies natürlich auch in Matrixschreibweise ausdrücken:
    \[ A \cdot x = \vect{62\\-2\\3} =: y \]
    Dabei ist $A$ gerade die Matrix aus Beispiel. % TODO: \ref
    
    Benutzen wir $A' = S\!AT$ (mit $S$ und $T$ auch wie in Beisp. %TODO: \ref)
    ), so sieht man, dass dies äquivalent zur Betrachtung von
    \[ A' \, \underbrace{ T^{-1} x }_{x'} = \underbrace{Sy}_{y'} \]
    ist. Wir können also die Lösungen $x'$ aus dem einfacheren Gleichungssystem
    $A' \cdot x' = y'$ ablesen und dann auf unser eigentliches Problem
    zurücktransformieren. Zunächst lesen wir die Lösungen des einfachen Systems
    ab:
    \[ A' \cdot x' = \Matrix{
                 1 &  0 &   0 \\
                 0 &  2 &   0 \\
                 0 &  0 &  42  } \cdot x' \overset{!}{=}
       \vect{3\\-4\\42} 
       = S \cdot \vect{62\\-2\\3}
       \quad
       \implies
       \quad
       x' = \vect{3\\-2\\1}  \]
    Nun berechnen wir mit Hilfe von $T$ die gesuchte Lösung $x$:
    \[ x = T x' = 
            \Matrix{
             0 &  1 &  7 \\
             0 &  0 &  1 \\
             1 &  0 & -7  } \cdot \vect{3\\-2\\1}
         = \vect{-9\\-1\\10} \]
    Setzt man nun $x_1 = -9,\; x_2 = -1,\; x_3 = 10$ in das obige
    Gleichungssystem ein, so wird man sehen, dass dies tatsächlich die
    Lösung darstellt.
\end{thBeisp}

Die war natürlich ein recht einfaches Beispiel. Da aber der Algorithmus zum
Elementarteilersatz komplett konstruktiv ist, kann man dieses Verfahren
natürlich auch auf beliebig komplexere lineare Diophantische Gleichungen
anwenden.

















\appendix
\bibliographystyle{plaindin}
\bibliography{bibsources}

\end{document}

