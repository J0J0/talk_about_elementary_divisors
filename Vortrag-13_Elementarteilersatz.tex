\documentclass[11pt,a4paper,ngerman,DIV=11]{scrreprt}
\usepackage[utf8]{inputenc}
\usepackage[T1]{fontenc}
\usepackage[ngerman]{babel}
\usepackage{amsmath}
\usepackage{amssymb}
\usepackage{amsthm}
\usepackage{mathtools}
\usepackage[babel]{csquotes}
\usepackage{enumerate} % TODO: replace with better 'enumitem'
\usepackage{ifmtarg}
%\usepackage[pdftex]{graphicx}
\usepackage[all]{xy}

\usepackage[pdftex,bookmarks,colorlinks=false,pdfborder={0 0 0},%
            pdftitle={Proseminar Modultheorie - Vortrag 13: Elementarteilersatz},%
            pdfauthor={Johannes Prem}]{hyperref}

%

\setlength{\parindent}{0pt}
\setlength{\parskip}{0.5em}

\newcommand{\origepsilon}{} % keine LaTeX-Kollisionen
\let\origepsilon=\epsilon
\let\epsilon=\varepsilon

\let\origphi=\phi
\let\phi=\varphi

\renewcommand{\qedsymbol}{$\blacksquare$}

\newtheoremstyle{mythms}
 {15pt}% Space above
 {12pt}% Space below 
 {}% Body font
 {}% Indent amount: Indent amount: empty = no indent, \parindent = normal paragraph indent
 {\bfseries}% Theorem head font
 {.}% Punctuation after theorem head
 {0.6cm}% Space after theorem head: Space after theorem head: { } = normal interword space; \newline = linebreak
 {}% Theorem head spec (can be left empty, meaning `normal')
 
\theoremstyle{mythms}
\newtheorem{globalnum}{DUMMY DUMMY DUMMY}
\newtheorem{thDef}[globalnum]{Definition}
\newtheorem{thSatz}[globalnum]{Satz}
\newtheorem{thPropos}[globalnum]{Proposition}
\newtheorem{thLemma}[globalnum]{Lemma}

\newtheorem{thBemerkung}[globalnum]{Bemerkung}
\newtheorem{thBeisp}[globalnum]{Beispiel}
\newtheorem{thBeispiele}[globalnum]{Beispiele}
\newenvironment{BspList}{\nopagebreak\begin{thBeispiele}\hfill\begin{itemize}}{\end{itemize}\end{thBeispiele}}
%

\DeclareMathOperator*{\Exists}{\exists}
\DeclareMathOperator*{\forAll}{\forall}

\newcommand{\Mid}{\,\middle\vert\,}

\makeatletter
\newcommand{\defmathbbsymbol}[1]{%
    \expandafter\newcommand\csname #1\endcsname[1][]{%
        \@ifmtarg{##1}{\mathbb{#1}}{\mathbb{#1}_{##1}}%
    }%
}
\makeatother

\defmathbbsymbol{N}
\defmathbbsymbol{Z}
\defmathbbsymbol{Q}
\defmathbbsymbol{R}
\defmathbbsymbol{C}

\DeclarePairedDelimiter{\abs}{\lvert}{\rvert}
\DeclarePairedDelimiter{\Spann}{\langle}{\rangle}

\DeclareMathOperator{\Kern}{ker}
\DeclareMathOperator{\modulelength}{\ell}

\newcommand{\barfrak}[1]{\bar{\mathfrak{#1}}}

\newcommand{\longto}{\longrightarrow}

\newcommand{\len}[1][R]{\modulelength_{#1}}


\setcounter{secnumdepth}{0}

\begin{document}


\chapter{Vorbereitungen}
Es sei $R$ im Folgenden immer ein kommutativer Ring. Damit ist auch klar, dass
wir nicht zwischen $R$-Links- und $R$-Rechtsmoduln unterscheiden müssen, so dass
wir einfach nur $R$-Modul schreiben werden.

\begin{thDef}[Länge eines Moduls]
    Sei $M$ ein $R$-Modul. Definiere nun $\len(M)$ als das Supremum der Längen
    $\ell$ von echt aufsteigenden Ketten von Untermoduln der Form:
    \[ 0 \subsetneq M_1 \subsetneq M_2 \subsetneq \cdots \subsetneq M_\ell = M \]
    Dabei lassen wir auch $\infty$ als Wert zu, so dass also $\len(M) \in
    \N\cup\{\infty\}$.
\end{thDef}

\begin{thBemerkung}
    Man kann die \emph{Länge eines Moduls} auch über nicht kommutativen Ringen
    einführen. Dies wird im Folgenden aber nicht benötigt.
\end{thBemerkung}

\begin{BspList}
\item
    Es ist offenbar $\len(M) = 0 \iff M = 0$.
\item
    Für den Sonderfall eines $n$-dim. $K$-Vektorraums $V$ (aufgefasst als
    $K$-Modul) hat man $\len[K](V) = n$, wie man schnell mit Hilfe des bekannen
    Isomorphismus $V \cong K^n$ und dessen Zerlegung $0\subsetneq K \subsetneq
    K^2 \subsetneq \cdots \subsetneq K^n$ sieht.
\item
    Für $\Z$ als $\Z$-Modul hat man die Länge $\infty$ und für $\ZRest2$ als
    $\ZRest2$-Modul erhält man die Länge $1$, da $\ZRest2$ keine echten
    Untermoduln besitzt.
\end{BspList}

Wir werden die Länge als Hilfsmittel brauchen, um den Elementarteilersatz
\pref{ets} zu zeigen. Dazu benötigen wir noch zwei einfache Lemma über die
Länge:

\begin{thLemma}[Länge des Restklassenmoduls über Hauptidealringen]%
    \label{lem:len:hir}
    Sei $R$ ein Hauptidealring\footnote{künftig abgekürzt durch: HIR} und $a\in
    R$ mit (bis auf Assoziiertheit eindeutiger) Primfaktorzerlegung 
    $a = p_1\cdots p_r$.
    % TODO: (Ex. nach \cite ..)
    Dann hat der Restklassenmodul $R/aR$ (über $R$) die Länge $\len(R/aR) = r$.
\end{thLemma}
\begin{proof}
    Zunächst ist klar, dass die Ideale in $R$ gerade den Untermoduln von $R$
    entsprechen, wenn man $R$ als Modul über sich selbst auffasst.

    Sei 
    \begin{align*}
        \pi\colon R &\to R/aR   \\
                  x &\mapsto \bar x
    \end{align*}
    der kanonische Epimorphismus. Dann kann man die Ideale $\barfrak{a} \subset
    R/aR$ mit Hilfe von $\pi$ unter der Bildung von $\pi^{-1}(\barfrak{a})$
    bijektiv den Idealen in $R$ zuordnen, die $aR$ enthalten. (nachprüfen!)
    
    Damit ist klar, dass man statt Idealketten des Typs
    \[ 0 \subsetneq \barfrak{a}_1 \subsetneq \barfrak{a}_2 \subsetneq 
        \cdots \subsetneq \barfrak{a}_\ell = R/aR      \]
    auch die folgenden Idealketten betrachten kann:
    \[ aR =: \mathfrak{a}_0 \subsetneq \mathfrak{a}_1 
                \subsetneq \mathfrak{a}_2 \subsetneq 
                \cdots \subsetneq \mathfrak{a}_\ell = R        \]
    Es genügt also, das Supremum der Längen derartiger Ketten zu betrachten. Da
    $R$ ein HIR ist, wird jedes dieser $\mathfrak{a}_i$ von einem Element $a_i$
    erzeugt: $\mathfrak{a}_i = \Spann{a_i}$. Da für $i\in\{0,\ldots \ell-1\}$ und
    die Ideale $\mathfrak{a}_i,\mathfrak{a}_{i+1}$ die Äquivalenz
    % TODO: \cite
    \[ \mathfrak{a}_i \subset \mathfrak{a}_{i+1} \iff a_{i+1} \mid a_i \]
    gilt und wir aber echte Inklusion betrachten, erhalten wir, dass auch
    $a_{i+1}$ ein echter Teiler von $a_i$ sein muss. Mit der vorausgesetzten
    Primfaktorzerlegung von $a$ sehen wir nun also leicht ein, dass wir gerade
    eine aufsteigende Kette von Idealen (die $aR$ enthalten) der Länge $r$
    bilden können.
    \\
\end{proof}

\begin{thLemma}[Additivität der Länge bei direkten Summen]%
    \label{lem:len:add}
    Sei ein $R$-Modul $M$ die direkte Summe zweier Untermoduln $M'$ und $M''$.
    Dann gilt:
    \[ \len(M) = \len(M') + \len(M'') \]
\end{thLemma}
\begin{proof}
    Wir zeigen die Gleichheit in zwei Schritten:
    \begin{description}
        \item[\enquote{\boldmath$\geq$}:]
            Hat man für die Untermoduln aufsteigende Ketten wie folgt:
            \begin{align*}
                0 \subsetneq M_1^{\prime\phantom\prime} \subsetneq
                            M_2^{\prime\phantom\prime} 
                            \subsetneq \cdots 
                            \subsetneq M_r^{\prime\phantom\prime}  &= M'    \\
                0 \subsetneq M_1'' \subsetneq M_2'' 
                            \subsetneq \cdots \subsetneq M_s'' &= M''
                ,
            \end{align*}
            so kann man direkt eine aufsteigende Kette der Länge $r+s$ für $M$
            angeben:
            \[ 0 \subsetneq M_1' \oplus 0 \subsetneq \cdots M_r' \oplus 0
                \subsetneq M_r' \oplus M_1'' \subsetneq M_r' \oplus M_s'' = M \]
            Es folgt: $\len(M) \geq \len(M') + \len(M'')$
            
        \item[\enquote{\boldmath$\leq$}:]
            Um diese Relation zu zeigen, betrachten wir eine aufsteigende Kette
            von Untermoduln von $M$:
            \[ 0 =: M_0 \subsetneq M_1 \subsetneq M_2 \subsetneq \cdots
            \subsetneq M_\ell = M \]
            Dann bekommen wir zu jedem $i\in\{0,\ldots,\ell-1\}$ ein anschauliches
            Diagramm wie folgt:
            \begin{equation*}
                \begin{xy}
                    \xymatrix@C=5pt@R=13pt{
                        M_i\cap M' \ar@{}[d]|-*[@][*1.2]{\subseteq}  & \subset
                        & M_i      \ar@{}[d]|-*[@][*1.2]{\subsetneq} & \overset\pi\longto
                        & \pi(M_i) \ar@{}[d]|-*[@][*1.2]{\subseteq}
                        \\
                        M_{i+1}\cap M' & \subset
                        & M_{i+1} & \overset\pi\longto
                        & \pi(M_{i+1})
                    }
                \end{xy}
            \end{equation*}
            Dabei bezeichne $\pi\colon M'\oplus M'' \to M''$ die Projektion auf
            $M''$. Wir bemerken dabei schon einmal, dass dann gerade 
            $\Kern(\pi) = M'$ ist.
            
            Nehmen wir nun einmal an, dass sowohl $M_i\cap M' = M_{i+1}\cap M'$
            als auch $\pi(M_i) = \pi(M_{i+1})$ in obigem Diagramm gilt.
            Wir zeigen nun, dass dies schon $M_i = M_{i+1}$ zur Folge hätte, was
            aber nach Voraussetzung ausgeschlossen ist.
            
            Sei also $x\in M_{i+1}$. Dann haben wir wegen der Gleichheit
            $\pi(M_i) = \pi(M_{i+1})$ ein $x'\in M_i$, so dass $\pi(x) =
            \pi(x')$ gilt. Da $\pi$ linear ist, also auch: $\pi(x-x') = 0 \iff
            x-x' \in \Kern(\pi) = M'$. Daraus folgt nun, dass $x$ und $x'$ auch
            in $M_{i+1}\cap M'$ sein müssen und wegen der Gleichheit 
            $M_i\cap M' = M_{i+1}\cap M'$ also auch in $M_i \cap M'$. Dann ist
            aber auch $x-x' \in M_i\cap M' \subset M_i$. Wenn aber $x'$ und
            $(x-x')$ in $M_i$ liegen, so auch $x = x' + (x-x') \in M_i$.
            Damit wäre $M_{i+1} \subset M_i$. Widerspruch.
            
            Also muss die Annahme falsch gewesen sein und es folgt:
            \[ \forAll_{i\in\{0,\ldots,\ell-1\}} \;
                M_i\cap M' \subsetneq M_{i+1}\cap M'  \;\vee\;
                \pi(M_i)   \subsetneq \pi(M_{i+1})  \]
            Daraus ergibt sich wie gewünscht:
            \[ \ell \leq \len(M') + \len(M'') \]
    \end{description}
    Insgesamt haben wir also Gleichheit, d.\,h. die Behauptung.
    \\
\end{proof}

\begin{thBemerkung}
    Kennt man das \emph{Fünferlemma}, so kann man den zweiten Schritt im
    vorangegangenen Beweis deutlich verkürzen, indem man einfach das folgende
    kommutative Diagramm mit exakten Zeilen betrachtet ($\pi$ wie oben):
    \begin{equation*}
        \newcommand{\xyhookarrowdown}{\ar@{^{(}->}[]-<0pt,11pt>;[d]+<0pt,11pt>}
        \begin{xy}
            \xymatrix@C=5pt@R=20pt{
                0                                   & \longto
                & M_i\cap M'    \xyhookarrowdown    & \longhookrightarrow
                & M_i           \xyhookarrowdown    & \overset\pi\longto
                & \pi(M_i)      \xyhookarrowdown    & \longto
                & 0
                \\
                0                                   & \longto
                & M_{i+1}\cap M'                    & \longhookrightarrow
                & M_{i+1}                           & \overset\pi\longto
                & \pi(M_{i+1})                      & \longto
                & 0
            }
        \end{xy}
    \end{equation*}
\end{thBemerkung}

Wir können nun zum eigentlichen Thema kommen:








\end{document}

