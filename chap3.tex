
\chapter{Beispiele und Anwendungen}
Ein erstes Beispiel haben wir schon bei Rümplein\cite{talk:ruemp} gesehen. Dort wurden
durch einfache Umformungen die Elementarteiler einer Matrix bestimmt:

\begin{thBeisp}[siehe auch {\cite[2.6]{talk:ruemp}}]
    Sei 
    \[ A := \Matrix{
           6 & 2 &  5 \\
          32 & 2 & 28 \\
          30 & 2 & 26  } \in \Z^{3\times3}
    \,. \]
    Mit dem beschriebenen Algorithmus erhält man dann:
    \[
        A
        \rightsquigarrow \ldots
        \rightsquigarrow 
    \Matrix{
        1 & 0 & 0 \\
        0 & 2 & 0 \\
        0 & 0 & 6  }
    \]
    Also sind die Elementarteiler: $1,2,6 \in\Z$.
\end{thBeisp}

Wir wollen uns daher gleich einem weiteren Beispiel zuwenden, in welchem wir
dann den Algorithmus aus dem Beweis zu \cref{ets:matrix} nochmal explizit
durchführen.

\begin{thBeisp}\label{bsp:mat}
    Betrachte den $\Z$-Modul $\Z^3$. Seien $v_1,v_2,v_3 \in\Z^3$ und der
    Untermodul $M\subset\Z^3$ wie folgt definiert:
    \begin{gather*}
        v_1 := \vect{2\\-2\\0}, \quad 
        v_2 := \vect{0\\ 0\\7}, \quad
        v_3 := \vect{8\\-2\\1},
        \\[2mm]
        M := \Spann[\big]{v_1,v_2,v_3}_{\Z}
    \end{gather*}
    Wir möchten nun die in \cref{ets} behauptete Basis $x_1,x_2,x_3$ von
    $\Z^3$ und diejenigen Elemente $\alpha_1,\alpha_2,\alpha_3\in\Z$ bestimmen, so dass
    $\alpha_1 x_1,\alpha_2 x_2,\alpha_3 x_3$ eine Basis von $M$ bilden.
    
    Dazu betrachten wir die lineare Abbildung, dargestellt durch die Matrix
    \[ A := \Matrix{ 2 & 0 &  8 \\
                    -2 & 0 & -2 \\
                     0 & 7 &  1  }  , \]
    welche offensichtlich die Standardbasis von $\Z^3$ auf das Erzeugendensystem
    $v_1,v_2,v_3$ von $M$ abbildet. Wir können nun das oben beschriebene
    Verfahren anwenden, um $A'$ zu erhalten. Diesmal sind wir aber zusätzlich an
    den Transformationsmatrizen $S$ und $T$ interessiert, so dass $S\!AT = A'$
    gilt. (Wieso wird weiter unten klar werden.) Um $S$ und $T$ gleich
    zusätzlich zu $A'$ zu erhalten, setzen wir $S_0:=T_0:=E_3$ (wobei $E_3$ die
    $(3\times3)$-Einheitsmatrix bezeichnet) und betrachten das erweiterte System
    $(S_0|A|T_0)$. Wir wenden dann alle Operationen, die Zeilen (Spalten)
    betreffen, zusätzlich auf $S_0$ ($T_0$) an bekommen so die gewünschten 
    Transformationsmatrizen gleich mit dazu.
    %
    (Dass dies funktioniert, wird ersichtlich, wenn man sich noch einmal klarmacht,
    dass eigentlich alle durchgeführten Operationen einer Matrixmultiplikation
    entsprechen: von links bei Zeilen- und von rechts bei Spaltenumformungen.)
    
    Es wird nun der Algorithmus für $A$ durchlaufen. Dabei stehen die nötigen
    Umformungen zwar immer nur bei der Matrix $A$, sie werden aber natürlich
    entsprechend auch auf $S_0$ und $T_0$ angewendet. Es ergibt sich also:
    
    \gmatrixright
    \setlength{\arrowpos}{5.2cm}
    \begin{equation*}
        \begin{trimatrixoperations}
             1 &  0 &  0 \\ 
             0 &  1 &  0 \\
             0 &  0 &  1
            \|
             2 &  0 &  8 \\
            -2 &  0 & -2 \\
             0 &  7 &  1
             \rowops
             \swap{0}{2}
             \colops
             \swap{0}{2}
            \|
             1 &  0 &  0 \\ 
             0 &  1 &  0 \\
             0 &  0 &  1
            %%%
            \nextstep
            %%%
             0 &  0 &  1 \\
             0 &  1 &  0 \\
             1 &  0 &  0  
            \|
             1 &  7 &  0 \\
            -2 &  0 & -2 \\
             8 &  0 &  2
             \colops
             \add[-7]{0}{1}
            \|
             0 &  0 &  1 \\
             0 &  1 &  0 \\
             1 &  0 &  0  
            %%%
            \nextstep
            %%%
             0 &  0 &  1 \\
             0 &  1 &  0 \\
             1 &  0 &  0  
            \|
             1 &   0 &  0 \\
            -2 &  14 & -2 \\
             8 & -56 &  2
             \rowops
             \add[2]{0}{1}
             \add[-8]{0}{2}
            \|
             0 &  0 &  1 \\
             0 &  1 &  0 \\
             1 & -7 &  0  
            %%%
            \nextstep
            %%%
             0 &  0 &  1 \\
             0 &  1 &  2 \\
             1 &  0 & -8  
            \|
            \vzfix
             1 &   0 &  0 \\
             0 &  14 & -2 \\
             0 & -56 &  2
             \colops
             \swap{1}{2}
            \|
             0 &  0 &  1 \\
             0 &  1 &  0 \\
             1 & -7 &  0  
            %%%
            \nextstep
            %%%
             0 &  0 &  1 \\
             0 &  1 &  2 \\
             1 &  0 & -8  
            \|
            \vzfix
             1 &  0 &   0 \\
             0 & -2 &  14 \\
             0 &  2 & -56
             \colops
             \add[7]{1}{2}
            \|
             0 &  1 &  0 \\
             0 &  0 &  1 \\
             1 &  0 & -7  
            %%%
            \nextstep
            %%%
             0 &  0 &  1 \\
             0 &  1 &  2 \\
             1 &  0 & -8  
            \|
            \vzfix
             1 &  0 &   0 \\
             0 & -2 &   0 \\
             0 &  2 & -42
             \rowops
             \add{1}{2}
            \|
             0 &  1 &  7 \\
             0 &  0 &  1 \\
             1 &  0 & -7  
        \end{trimatrixoperations}
    \end{equation*}
    \begin{equation*}
        \begin{trimatrixoperations}
            %%%
            \continued
            %%%
             0 &  0 &  1 \\
             0 &  1 &  2 \\
             1 &  1 & -6  
            \|
            \vzfix
             1 &  0 &   0 \\
             0 & -2 &   0 \\
             0 &  0 & -42
             \rowops
             \mult{1}{\cdot(-1)}
             \colops
             \mult{2}{\cdot(-1)}
            \|
             0 &  1 &  7 \\
             0 &  0 &  1 \\
             1 &  0 & -7  
            %%%
            \nextstep
            %%%
             0 &  0 &  1 \\
             0 & -1 & -2 \\
             1 &  1 & -6  
            \|
            \vzfix
             1 &  0 &   0 \\
             0 &  2 &   0 \\
             0 &  0 &  42
            \|
             0 &  1 & -7 \\
             0 &  0 & -1 \\
             1 &  0 &  7  
        \end{trimatrixoperations}
    \end{equation*}
    
    \medskip
    Wir erhalten also die Elementarteiler $\alpha_1 = 1,\; \alpha_2 = 2$ und
    $\alpha_3 = 42$ und zusätzlich die Transformationsmatrizen
    \[ S = \Matrix{
             0 &  0 &  1 \\
             0 & -1 & -2 \\
             1 &  1 & -6  } 
       \quad\text{und}\quad
       T = \Matrix{
             0 &  1 & -7 \\
             0 &  0 & -1 \\
             1 &  0 &  7  }
    . \]
    (Streng genommen haben wir im letzten Schritt der obigen Umformungen
    außerhalb des beschriebenen Verfahrens operiert; dass die Multiplikation
    einer Zeile oder Spalte mit einer Einheit auch erlaubt ist, sieht man aber
    schnell ein.)
    
    Nun ist aber $S$ gerade die Basiswechselmatrix von der neuen Basis
    $x_1,x_2,x_3$ in die Standardbasis und wir müssen somit nur noch $S$
    invertieren und können dann die passenden Elemente $x_1,x_2,x_3$ in den Spalten
    von $S^{-1}$ ablesen. (Es bleibt dem Leser überlassen, ein
    Invertierungsverfahren konkret anzuwenden; es bietet sich beispielsweise
    \cite[2.6]{talk:loh} an.) Man bekommt dann:
    \[ S^{-1} = \Matrix{
                     8 &  1 & 1 \\
                    -2 & -1 & 0 \\
                     1 &  0 & 0  }  \]
    Da nun $S^{-1}$ gerade von der Standardbasis in die neue Basis wechselt,
    haben wir für letztere:
    \[
        x_1 := \vect{8\\-2\\1}, \quad 
        x_2 := \vect{1\\-1\\0}, \quad
        x_3 := \vect{1\\ 0\\0}    \]
    Da wir die Elementarteiler schon haben, können wir nun auch sofort die
    gewünscht Basis für $M$ angeben:
    \[
        \alpha_1 x_1 := \vect{8\\-2\\1}, \quad 
        \alpha_2 x_2 := \vect{2\\-2\\0}, \quad
        \alpha_3 x_3 := \vect{42\\0\\0}    \]

    Wir haben nun zwar $T$ gar nicht benötigt, jedoch wird im nächsten Beispiel 
    klar werden, warum wir uns die Mühe gemacht haben, auch $T$ zu berechnen.
\end{thBeisp}

Kommen wir nun zu einer sehr anschaulichen Anwendung des Elementarteilersatzes.
Man kann diesen nämlich auch dazu verwenden, sog. \emph{lineare Diophantische
Gleichungen} konstruktiv zu lösen. Darunter versteht man lineare
Gleichungssysteme mit Koeffizienten aus~$\Z$, wobei man auch nur Lösungen aus
$\Z$ sucht. Konkret:

\begin{thBeisp}
    Wir suchen ganzzahlige Lösungen des folgenden Gleichungssystems:
    \begin{alignat*}{4}
         &2x_1  &\quad&      &\quad&            +8x_3  &\quad=&\quad \; 62             \\
        -&2x_1  &\quad&      &\quad&            -2x_3  &\quad=&\quad            {-2}   \\
         &      &\quad& 7x_2 &\quad& +\phantom{1} x_3  &\quad=&\quad \phantom{+}  3
    \end{alignat*}
    %
    Wir können dies natürlich auch in Matrixschreibweise ausdrücken:
    \[ A \cdot x = \vect{62\\-2\\3} =: y \]
    Dabei ist $A$ gerade die Matrix aus \cref{bsp:mat}.
    
    Benutzen wir $A' = S\!AT$ (mit $S$ und $T$ auch wie in \cref{bsp:mat}), so
    sieht man, dass dies äquivalent zur Betrachtung von
    \[ A' \, \underbrace{ T^{-1} x \vphantom{y} }_{x'} = \underbrace{Sy}_{y'} \]
    ist. Wir können also die Lösungen $x'$ aus dem einfacheren Gleichungssystem
    $A' \cdot x' = y'$ ablesen und dann auf unser eigentliches Problem
    zurücktransformieren. Zunächst lesen wir die Lösungen des einfachen Systems
    ab:
    \[ A' \cdot x' = \Matrix{
                 1 &  0 &   0 \\
                 0 &  2 &   0 \\
                 0 &  0 &  42  } \cdot x' \overset{!}{=}
       \vect{3\\-4\\42} 
       = S \cdot \vect{62\\-2\\3}
       \quad
       \implies
       \quad
       x' = \vect{3\\-2\\1}  \]
    Nun berechnen wir mit Hilfe von $T$ die gesuchte Lösung $x$:
    \[ x = T x' = 
            \Matrix{
             0 &  1 &  7 \\
             0 &  0 &  1 \\
             1 &  0 & -7  } \cdot \vect{3\\-2\\1}
         = \vect{-9\\-1\\10} \]
    Setzt man nun $x_1 = -9,\; x_2 = -1,\; x_3 = 10$ in das obige
    Gleichungssystem ein, so wird man sehen, dass dies tatsächlich die
    Lösung darstellt.
\end{thBeisp}

Dies war natürlich ein recht einfaches Beispiel. Da aber der Algorithmus zum
Elementarteilersatz komplett konstruktiv ist, kann man dieses Verfahren
natürlich auch auf beliebig komplexere lineare Diophantische Gleichungen
anwenden.

Bisher hatten wir immer nur den Ring $\Z$ betrachtet. Der beschriebene
Algorithmus funktioniert aber natürlich auch über komplizierteren HIR, was nun
noch an folgendem abschließenden Beispiel demonstriert werden soll.

\begin{thBeisp}
    Betrachte den Polynomring $\R[X]$ über den reellen Zahlen. Dieser ist ein
    euklidischer Ring (siehe \cite[??]{talk:rief}),
    wobei wir als Gradfunktion gerade die Abbildung $\deg\colon\R[X]\to\N$
    haben, die jedem Polynom seinen Grad zuordnet. Weiter haben wir mit der
    Polynomdivision ein praktisches Mittel, um zu $f,g\in\R[X]$ diejenigen
    $q,r\in\R[X]$ zu bestimmen, so dass $f = qg+r$ gilt, mit: $r=0$ oder
    $\deg(r)<\deg(g)$.
    
    Sei nun also
    \[ A := \Matrix{ X      &  2X   \\
                     0      &  X+1  \\
                     X^2-X  &  0     }  
    \in \bigl(\R[X]\bigr)^{3\times2}
    \]
    eine Matrix mit Koeffizienten aus $\R[X]$. Wir wollen noch einmal den
    Algorithmus für die Bestimmung der Elementarteiler inklusive der
    Transformationsmatrizen durchlaufen:
    
    \gmatrixcenter
    \setlength{\arrowpos}{6.4cm}
    \begin{equation*}
        \begin{trimatrixoperations}[C]
            1 & 0 & 0 \\
            0 & 1 & 0 \\
            0 & 0 & 1 
            \|
            X     &  2X   \\
            0     &  X+1  \\
            X^2-X &  0
            \colops
            \add[-2]{0}{1}
            \|
            1 & 0 \\
            0 & 1
            %%%
            \nextstep
            %%%
            1 & 0 & 0 \\
            0 & 1 & 0 \\
            0 & 0 & 1 
            \|
            X     &  0    \\
            0     &  X+1  \\
            X^2-X &  -2X^2+2X
            \rowops
            \add[-(X-1)]{0}{2}
            \|
            1 & -2 \\
            0 & 1
            %%%
            \nextstep
            %%%
            1 & 0 & 0 \\
            0 & 1 & 0 \\
            -X+1 & 0 & 1 
            \|
            X     &  0    \\
            0     &  X+1  \\
            0 &  -2X^2+2X
            \rowops
            \add{1}{0}
            \|
            1 & -2 \\
            0 & 1
            %%%
            \nextstep
            %%%
            1 & 1 & 0 \\
            0 & 1 & 0 \\
            -X+1 & 0 & 1 
            \|
            X     &  X+1  \\
            0     &  X+1  \\
            0 &  -2X^2+2X
            \colops
            \add[-1]{0}{1}
            \|
            1 & -2 \\
            0 & 1
            %%%
            \nextstep
            %%%
            1 & 1 & 0 \\
            0 & 1 & 0 \\
            -X+1 & 0 & 1 
            \|
            X     &    1  \\
            0     &  X+1  \\
            0 &  -2X^2+2X
            \colops
            \swap{0}{1}
            \|
            1 & -3 \\
            0 & 1
            %%%
            \nextstep
            %%%
            1 & 1 & 0 \\
            0 & 1 & 0 \\
            -X+1 & 0 & 1 
            \|
              1        &   X   \\
            X+1        &   0   \\
            -2X^2+2X   &   0  
            \colops
            \add[-X]{0}{1}
            \|
            -3 & 1 \\
            1  & 0
        \end{trimatrixoperations}
    \end{equation*}
    \setlength{\arrowpos}{8.4cm}
    \begin{equation*}
        \begin{trimatrixoperations}[C]
            %%%
            \continued
            %%%
            1 & 1 & 0 \\
            0 & 1 & 0 \\
            -X+1 & 0 & 1 
            \|
              1        &   0        \\
            X+1        &   -X^2-X   \\
            -2X^2+2X   &   2X^3-2X^2  
            \rowops
            \add[-(X+1)]{0}{1}
            \|
            -3 & 1+3X \\
            1  & -X
            %%%
            \nextstep
            %%%
            1 & 1 & 0 \\
            -X-1 & -X & 0 \\
            -X+1 & 0 & 1 
            \|
              1        &   0        \\
              0        &   -X^2-X   \\
            -2X^2+2X   &   2X^3-2X^2  
            \rowops
            \add[-(-2X^2+2X)]{0}{2}
            \|
            -3 & 1+3X \\
            1  & -X
        \end{trimatrixoperations}
    \end{equation*}
    \setlength{\arrowpos}{8.4cm}
    \begin{equation*}
        \begin{trimatrixoperations}[C]
            %%%
            \continued
            %%%
            1 & 1 & 0 \\
            -X-1 & -X & 0 \\
            2X^2-3X+1 & 2X^2-2X & 1 
            \|
              1        &   0        \\
              0        &   -X^2-X   \\
              0        &   2X^3-2X^2  
            \rowops
            \add[2X-4]{1}{2}
            \|
            -3 & 1+3X \\
            1  & -X
            %%%
            \nextstep
            %%%
            1 & 1 & 0       \\
            -X-1 & -X & 0   \\
            -X+5 & 2X & 1 
            \|
              1        &   0        \\
              0        &   -X^2-X   \\
              0        &   4X  
            \rowops
            \swap{1}{2}
            \|
            -3 & 1+3X \\
            1  & -X
            %%%
            \nextstep
            %%%
            1 & 1 & 0       \\
            -X+5 & 2X & 1   \\
            -X-1 & -X & 0   
            \|
              1        &   0        \\
              0        &   4X       \\
              0        &   -X^2-X
            \rowops
            \add[\frac{1}{4}(X+1)]{1}{2}
            \|
            -3 & 1+3X \\
            1  & -X
            %%%
            \nextstep
            %%%
            1 & 1 & 0       \\
            -X+5 & 2X & 1   \\
            -\frac{1}{4} (X^2-1) & \frac{1}{2} (X^2-X) & \frac{1}{4} (X+1)   
            \|
              1        &   0        \\
              0        &   4X       \\
              0        &   0
            \colops
            \mult{1}{\cdot 4^{-1}}
            \|
            -3 & 1+3X \\
            1  & -X
            %%%
            \nextstep
            %%%
            1 & 1 & 0       \\
            -X+5 & 2X & 1   \\
            -\frac{1}{4} (X^2-1) & \frac{1}{2} (X^2-X) & \frac{1}{4} (X+1)   
            \|
              1        &   0       \\
              0        &   X       \\
              0        &   0
            \|
            -3 & \frac{1}{4}+\frac{3}{4}X \\
            1  & -\frac{1}{4}X
        \end{trimatrixoperations}
    \end{equation*}
    
    \medskip
    Wir erhalten also die Elementarteiler $1,X\in\R[X]$ und die
    Transformationsmatrizen
    \begin{align*}
        S &= \Matrix{
        1 & 1 & 0       \\[3pt]
        -X+5 & 2X & 1   \\[3pt]
        -\frac{1}{4} (X^2-1) & \frac{1}{2} (X^2-X) & \frac{1}{4} (X+1)   }
        \in \bigl(\R[X]\bigr)^{3\times3}
        \qquad\text{und}
        \\[3mm]
        T &= \Matrix{
            -3 & \frac{1}{4}+\frac{3}{4}X   \\[3pt]
            1  & -\frac{1}{4}X                  }
        \in \bigl(\R[X]\bigr)^{2\times2}
    .
    \end{align*}
        
    
\end{thBeisp}














