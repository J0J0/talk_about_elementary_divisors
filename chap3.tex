
\chapter{Beispiele und Anwendungen}
Ein erstes Beispiel haben wir schon in [TODO] % TODO: \cite 08
gesehen. Dort wurden durch einfache Umformungen die Elementarteiler einer Matrix
bestimmt:

\begin{thBeisp}[siehe auch {[TODO]}] % TODO: \cite 08/2.6
    Sei 
    \[ A := \Matrix{
           6 & 2 &  5 \\
          32 & 2 & 28 \\
          30 & 2 & 26  } \in \Z^{3\times3}
    \,. \]
    Mit dem beschriebenen Algorithmus erhält man dann:
    \[
        A
        \rightsquigarrow \ldots
        \rightsquigarrow 
    \Matrix{
        1 & 0 & 0 \\
        0 & 2 & 0 \\
        0 & 0 & 6  }
    \]
    Also sind die Elementarteiler: $1,2,6 \in\Z$.
\end{thBeisp}

Wir wollen uns daher gleich einem weiteren Beispiel zuwenden, in welchem
wir dann den Algorithmus aus dem Beweis zu Lemma % TODO: \ref
nochmal explizit durchführen.

\begin{thBeisp}
    Betrachte den $\Z$-Modul $\Z^3$. Seien $v_1,v_2,v_3 \in\Z^3$ und der
    Untermodul $M\subset\Z^3$ wie folgt definiert:
    \begin{gather*}
        v_1 := \vect{2\\-2\\0}, \quad 
        v_2 := \vect{0\\ 0\\7}, \quad
        v_3 := \vect{8\\-2\\1},
        \\[2mm]
        M := \Spann[\big]{v_1,v_2,v_3}_{\Z}
    \end{gather*}
    Wir möchten nun die in Satz % TODO: \ref
    behauptete Basis $x_1,\ldots,x_3$ von $\Z^3$ und Elemente
    $\alpha_1,\alpha_2,\alpha_3\in\Z$ bestimmen, so dass 
    $\alpha_1 x_1,\alpha_2 x_2,\alpha_3 x_3$ eine Basis von $M$ bilden.
    
    Dazu betrachten wir die lineare Abbildung, dargestellt durch die Matrix
    \[ A := \Matrix{ 2 & 0 &  8 \\
                    -2 & 0 & -2 \\
                     0 & 7 &  1  }  , \]
    welche offensichtlich die Standardbasis von $\Z^3$ auf das Erzeugendensystem
    $v_1,v_2,v_3$ von $M$ abbildet. Wir können nun das oben beschriebene
    Verfahren anwenden, um $A'$ zu erhalten. Diesmal sind wir aber zusätzlich an
    den Transformationsmatrizen $S$ und $T$ interessiert, so dass $S\!AT = A'$
    gilt. (Wieso wird weiter unten klar werden.) Um $S$ und $T$ gleich
    zusätzlich zu $A'$ zu erhalten, setzen wir $S_0:=T_0:=E_3$ (wobei $E_3$ die
    $(3\times3)$-Einheitsmatrix bezeichnet) und betrachten das erweiterte System
    $(S_0|A|T_0)$. Wir wenden dann alle Operationen, die Zeilen (Spalten)
    betreffen, zusätzlich auf $S_0$ ($T_0$) an bekommen so die gewünschten 
    Transformationsmatrizen gleich mit dazu.
    %
    (Dass dies funktioniert, wird ersichtlich, wenn man sich noch einmal klarmacht,
    dass eigentlich alle durchgeführten Operationen einer Matrixmultiplikation
    entsprechen: von links bei Zeilen- und von recht bei Spaltenumformungen.)
    
    Es wird nun der Algorithmus für $A$ durchlaufen. Dabei stehen die nötigen
    Umformungen zwar immer nur bei der Matrix $A$, sie werden aber natürlich
    entsprechend auch auf $S_0$ und $T_0$ angewendet. Es ergibt sich also:
    
    \begin{equation*}
        \begin{trimatrixoperations}
             1 &  0 &  0 \\ 
             0 &  1 &  0 \\
             0 &  0 &  1
            \|
             2 &  0 &  8 \\
            -2 &  0 & -2 \\
             0 &  7 &  1
             \rowops
             \swap{0}{2}
             \colops
             \swap{0}{2}
            \|
             1 &  0 &  0 \\ 
             0 &  1 &  0 \\
             0 &  0 &  1
            %%%
            \nextstep
            %%%
             0 &  0 &  1 \\
             0 &  1 &  0 \\
             1 &  0 &  0  
            \|
             1 &  7 &  0 \\
            -2 &  0 & -2 \\
             8 &  0 &  2
             \colops
             \add[-7]{0}{1}
            \|
             0 &  0 &  1 \\
             0 &  1 &  0 \\
             1 &  0 &  0  
            %%%
            \nextstep
            %%%
             0 &  0 &  1 \\
             0 &  1 &  0 \\
             1 &  0 &  0  
            \|
             1 &   0 &  0 \\
            -2 &  14 & -2 \\
             8 & -56 &  2
             \rowops
             \add[2]{0}{1}
             \add[-8]{0}{2}
            \|
             0 &  0 &  1 \\
             0 &  1 &  0 \\
             1 & -7 &  0  
            %%%
            \nextstep
            %%%
             0 &  0 &  1 \\
             0 &  1 &  2 \\
             1 &  0 & -8  
            \|
            \vzfix
             1 &   0 &  0 \\
             0 &  14 & -2 \\
             0 & -56 &  2
             \colops
             \swap{1}{2}
            \|
             0 &  0 &  1 \\
             0 &  1 &  0 \\
             1 & -7 &  0  
            %%%
            \nextstep
            %%%
             0 &  0 &  1 \\
             0 &  1 &  2 \\
             1 &  0 & -8  
            \|
            \vzfix
             1 &  0 &   0 \\
             0 & -2 &  14 \\
             0 &  2 & -56
             \colops
             \add[7]{1}{2}
            \|
             0 &  1 &  0 \\
             0 &  0 &  1 \\
             1 &  0 & -7  
            %%%
            \nextstep
            %%%
             0 &  0 &  1 \\
             0 &  1 &  2 \\
             1 &  0 & -8  
            \|
            \vzfix
             1 &  0 &   0 \\
             0 & -2 &   0 \\
             0 &  2 & -42
             \rowops
             \add{1}{2}
            \|
             0 &  1 &  7 \\
             0 &  0 &  1 \\
             1 &  0 & -7  
        \end{trimatrixoperations}
    \end{equation*}
    \begin{equation*}
        \begin{trimatrixoperations}
            %%%
            \continued
            %%%
             0 &  0 &  1 \\
             0 &  1 &  2 \\
             1 &  1 & -6  
            \|
            \vzfix
             1 &  0 &   0 \\
             0 & -2 &   0 \\
             0 &  0 & -42
             \rowops
             \mult{1}{\cdot(-1)}
             \colops
             \mult{2}{\cdot(-1)}
            \|
             0 &  1 &  7 \\
             0 &  0 &  1 \\
             1 &  0 & -7  
            %%%
            \nextstep
            %%%
             0 &  0 &  1 \\
             0 & -1 & -2 \\
             1 &  1 & -6  
            \|
            \vzfix
             1 &  0 &   0 \\
             0 &  2 &   0 \\
             0 &  0 &  42
            \|
             0 &  1 & -7 \\
             0 &  0 & -1 \\
             1 &  0 &  7  
        \end{trimatrixoperations}
    \end{equation*}
    
    \medskip
    Wir erhalten also die Elementarteiler $\alpha_1 = 1,\; \alpha_2 = 2$ und
    $\alpha_3 = 42$ und zusätzlich die Transformationsmatrizen
    \[ S = \Matrix{
             0 &  0 &  1 \\
             0 & -1 & -2 \\
             1 &  1 & -6  } 
       \quad\text{und}\quad
       T = \Matrix{
             0 &  1 & -7 \\
             0 &  0 & -1 \\
             1 &  0 &  7  }
    . \]
    (Streng genommen haben wir im letzten Schritt der obigen Umformungen
    außerhalb des beschriebenen Verfahrens operiert; dass die Multiplikation
    einer Zeile oder Spalte mit einer Einheit auch erlaubt ist, sieht man aber
    schnell ein.)
    
    Nun ist aber $S$ gerade die Basiswechselmatrix von der neuen Basis
    $x_1,x_2,x_3$ in die Standardbasis und wir müssen somit nur noch $S$
    invertieren und können dann die passenden Elemente $x_1,x_2,x_3$ in den Spalten
    von $S^{-1}$ ablesen. (Es bleibt dem Leser überlassen, ein
    Invertierungsverfahren konkret anzuwenden; es bietet sich beispielsweise
    [Loher] % TODO: \cite 10/2.8
    an.) Man bekommt dann:
    \[ S^{-1} = \Matrix{
                     8 &  1 & 1 \\
                    -2 & -1 & 0 \\
                     1 &  0 & 0  }  \]
    Da nun $S^{-1}$ gerade von der Standardbasis in die neue Basis wechselt,
    haben wir für letztere:
    \[
        x_1 := \vect{8\\-2\\1}, \quad 
        x_2 := \vect{1\\-1\\0}, \quad
        x_3 := \vect{1\\ 0\\0}    \]
    Da wir die Elementarteiler schon haben, können wir nun auch sofort die
    gewünscht Basis für $M$ angeben:
    \[
        \alpha_1 x_1 := \vect{8\\-2\\1}, \quad 
        \alpha_2 x_2 := \vect{2\\-2\\0}, \quad
        \alpha_3 x_3 := \vect{42\\0\\0}    \]

    Wir haben nun zwar $T$ gar nicht benötigt, jedoch wird im nächsten Beispiel 
    klar werden, warum wir uns die Mühe gemacht haben, auch $T$ zu berechnen.
\end{thBeisp}

Kommen wir nun zu einer sehr anschaulichen Anwendung des Elementarteilersatzes.
Man kann diesen nämlich auch dazu verwenden, sog. \emph{lineare Diophantische
Gleichungen} konstruktiv zu lösen. Darunter versteht man lineare
Gleichungssysteme mit Koeffizienten aus~$\Z$, wobei man auch nur Lösungen aus
$\Z$ sucht. Konkret:

\begin{thBeisp}
    Wir suchen ganzzahlige Lösungen des folgenden Gleichungssystems:
    \begin{alignat*}{4}
         &2x_1  &\quad&      &\quad&            +8x_3  &\quad=&\quad \; 62             \\
        -&2x_1  &\quad&      &\quad&            -2x_3  &\quad=&\quad            {-2}   \\
         &      &\quad& 7x_2 &\quad& +\phantom{1} x_3  &\quad=&\quad \phantom{+}  3
    \end{alignat*}
    %
    Wir können dies natürlich auch in Matrixschreibweise ausdrücken:
    \[ A \cdot x = \vect{62\\-2\\3} =: y \]
    Dabei ist $A$ gerade die Matrix aus Beispiel. % TODO: \ref
    
    Benutzen wir $A' = S\!AT$ (mit $S$ und $T$ auch wie in Beisp. %TODO: \ref)
    ), so sieht man, dass dies äquivalent zur Betrachtung von
    \[ A' \, \underbrace{ T^{-1} x }_{x'} = \underbrace{Sy}_{y'} \]
    ist. Wir können also die Lösungen $x'$ aus dem einfacheren Gleichungssystem
    $A' \cdot x' = y'$ ablesen und dann auf unser eigentliches Problem
    zurücktransformieren. Zunächst lesen wir die Lösungen des einfachen Systems
    ab:
    \[ A' \cdot x' = \Matrix{
                 1 &  0 &   0 \\
                 0 &  2 &   0 \\
                 0 &  0 &  42  } \cdot x' \overset{!}{=}
       \vect{3\\-4\\42} 
       = S \cdot \vect{62\\-2\\3}
       \quad
       \implies
       \quad
       x' = \vect{3\\-2\\1}  \]
    Nun berechnen wir mit Hilfe von $T$ die gesuchte Lösung $x$:
    \[ x = T x' = 
            \Matrix{
             0 &  1 &  7 \\
             0 &  0 &  1 \\
             1 &  0 & -7  } \cdot \vect{3\\-2\\1}
         = \vect{-9\\-1\\10} \]
    Setzt man nun $x_1 = -9,\; x_2 = -1,\; x_3 = 10$ in das obige
    Gleichungssystem ein, so wird man sehen, dass dies tatsächlich die
    Lösung darstellt.
\end{thBeisp}

Die war natürlich ein recht einfaches Beispiel. Da aber der Algorithmus zum
Elementarteilersatz komplett konstruktiv ist, kann man dieses Verfahren
natürlich auch auf beliebig komplexere lineare Diophantische Gleichungen
anwenden.















