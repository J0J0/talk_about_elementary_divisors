
\chapter{Beispiele und Anwendungen}
Ein erstes Beispiel haben wir schon in [TODO] % TODO: \cite 08
gesehen. Dort wurden durch einfache Umformungen die Elementarteiler einer Matrix
bestimmt:

\begin{thBeisp}[siehe auch {[TODO]}] % TODO: \cite 08/2.6
    Sei 
    \[ A := \Matrix{
           6 & 2 &  5 \\
          32 & 2 & 28 \\
          30 & 2 & 26  } \in \Z^{3\times3}
    \,. \]
    Mit dem beschriebenen Algorithmus erhält man dann:
    \[
        A
        \rightsquigarrow \ldots
        \rightsquigarrow 
    \Matrix{
        1 & 0 & 0 \\
        0 & 2 & 0 \\
        0 & 0 & 6  }
    \]
    Also sind die Elementarteiler: $1,2,6 \in\Z$.
\end{thBeisp}

Wir wollen uns daher gleich einem weiteren Beispiel zuwenden, in welchem
wir dann den Algorithmus aus dem Beweis zu Lemma % TODO: \ref
nochmal explizit durchführen.

\begin{thBeisp}
    Betrachte den $\Z$-Modul $\Z^3$. Seien $v_1,v_2,v_3 \in\Z^3$ und der
    Untermodul $M\subset\Z^3$ wie folgt definiert:
    \begin{gather*}
        v_1 := \vect{2\\-2\\0}, \quad 
        v_2 := \vect{0\\ 0\\7}, \quad
        v_3 := \vect{8\\-2\\1},
        \\[2mm]
        M := \Spann[\big]{v_1,v_2,v_3}_{\Z}
    \end{gather*}
    Wir möchten nun die in Satz % TODO: \ref
    behauptete Basis $x_1,\ldots,x_3$ von $\Z^3$ und Elemente
    $\alpha_1,\ldots,\alpha_3\in\Z$ bestimmen, so dass 
    $\alpha_1 x_1,\ldots,\alpha_3 x_3$ eine Basis von $M$ bilden.
    
    Dazu betrachten wir die lineare Abbildung, dargestellt durch die Matrix
    \[ A := \Matrix{ 2 & 0 &  8 \\
                    -2 & 0 & -2 \\
                     0 & 7 &  1  }  \, \]
    welche offensichtlich die Standardbasis von $\Z^3$ auf das Erzeugendensystem
    $v_1,v_2,v_3$ von $M$ abbildet. Wir können nun das oben beschriebene
    Verfahren anwenden, um $A'$ zu erhalten. Diesmal sind wir aber zusätzlich an
    den Transformationsmatrizen $S$ und $T$ interessiert, so dass $SAT = A'$
    gilt. (Wieso wird weiter unten klar werden.) Um $S$ und $T$ gleich
    zusätzlich zu $A'$ zu erhalten, setzen wir $S_0:=T_0:=E_3$ (wobei $E_3$ die
    $(3\times3)$-Einheitsmatrix bezeichnet) und betrachten das erweiterte System
    $S_0|A|T_0$. Wir wenden dann alle Operationen, die Zeilen (Spalten)
    betreffen, zusätzlich auf $S_0$ ($T_0$) an bekommen so die gewünschten 
    Transformationsmatrizen gleich mit dazu.
    %
    (Dass dies funktioniert, wird ersichtlich, wenn man sich noch einmal klarmacht,
    dass eigentlich alle durchgeführten Operationen einer Matrixmultiplikation
    entsprechen: von links bei Zeilen- und von recht bei Spaltenumformungen.)
    
    Es wird nun der Algorithmus für $A$ durchlaufen. Dabei stehen die nötigen
    Umformungen zwar immer nur bei der Matrix $A$, sie werden aber natürlich
    entsprechend auch auf $S_0$ und $T_0$ angewendet.
    
    \begin{equation*}
        \begin{matrixoperations}
             1 &  0 &  0 \\ 
             0 &  1 &  0 \\
             0 &  0 &  1
            \|
             2 &  0 &  8 \\
            -2 &  0 & -2 \\
             0 &  7 &  1
             \rowops
             \swap{0}{2}
             \colops
             \swap{0}{2}
            \|
             1 &  0 &  0 \\ 
             0 &  1 &  0 \\
             0 &  0 &  1
            %%%
            \nextstep
            %%%
             0 &  0 &  1 \\
             0 &  1 &  0 \\
             1 &  0 &  0  
            \|
             1 &  7 &  0 \\
            -2 &  0 & -2 \\
             8 &  0 &  2
             \colops
             \add[-7]{0}{1}
            \|
             0 &  0 &  1 \\
             0 &  1 &  0 \\
             1 &  0 &  0  
            %%%
            \nextstep
            %%%
             0 &  0 &  1 \\
             0 &  1 &  0 \\
             1 &  0 &  0  
            \|
             1 &   0 &  0 \\
            -2 &  14 & -2 \\
             8 & -56 &  2
             \rowops
             \add[2]{0}{1}
             \add[-8]{0}{2}
            \|
             0 &  0 &  1 \\
             0 &  1 &  0 \\
             1 & -7 &  0  
            %%%
            \nextstep
            %%%
             0 &  0 &  1 \\
             0 &  1 &  2 \\
             1 &  0 & -8  
            \|
            \vzfix
             1 &   0 &  0 \\
             0 &  14 & -2 \\
             0 & -56 &  2
             \colops
             \swap{1}{2}
            \|
             0 &  0 &  1 \\
             0 &  1 &  0 \\
             1 & -7 &  0  
            %%%
            \nextstep
            %%%
             0 &  0 &  1 \\
             0 &  1 &  2 \\
             1 &  0 & -8  
            \|
            \vzfix
             1 &  0 &   0 \\
             0 & -2 &  14 \\
             0 &  2 & -56
             \colops
             \add[7]{1}{2}
            \|
             0 &  1 &  0 \\
             0 &  0 &  1 \\
             1 &  0 & -7  
            %%%
            \nextstep
            %%%
             0 &  0 &  1 \\
             0 &  1 &  2 \\
             1 &  0 & -8  
            \|
            \vzfix
             1 &  0 &   0 \\
             0 & -2 &   0 \\
             0 &  2 & -42
             \rowops
             \add{1}{2}
            \|
             0 &  1 &  7 \\
             0 &  0 &  1 \\
             1 &  0 & -7  
        \end{matrixoperations}
    \end{equation*}
    \begin{equation*}
        \begin{matrixoperations}
            %%%
            \continued
            %%%
             0 &  0 &  1 \\
             0 &  1 &  2 \\
             1 &  1 & -6  
            \|
            \vzfix
             1 &  0 &   0 \\
             0 & -2 &   0 \\
             0 &  0 & -42
             \rowops
             \mult{1}{\cdot(-1)}
             \colops
             \mult{2}{\cdot(-1)}
            \|
             0 &  1 &  7 \\
             0 &  0 &  1 \\
             1 &  0 & -7  
            %%%
            \nextstep
            %%%
             0 &  0 &  1 \\
             0 & -1 & -2 \\
             1 &  1 & -6  
            \|
            \vzfix
             1 &  0 &   0 \\
             0 &  2 &   0 \\
             0 &  0 &  42
            \|
             0 &  1 & -7 \\
             0 &  0 & -1 \\
             1 &  0 &  7  
        \end{matrixoperations}
    \end{equation*}
\end{thBeisp}















