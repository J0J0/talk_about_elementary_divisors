
\chapter{Vorwort und Notation}
Dieses Skript soll den Inhalt von Bosch\cite[S.\,206\,ff., 6.3]{bookc:bosch08}
zusammenfassen. Deshalb werden mehrfach Beweise nicht im Detail ausgeführt und
dabei auf \cite{bookc:bosch08} verwiesen.

Weiter sei angemerkt, dass der Beweis des Elementarteilersatzes in
Bosch\cite{bookc:bosch08} sehr \enquote{intuitiv} ist, so dass er zwar leicht zu
verstehen ist, mehrfach jedoch eher mathematisch unvollständig erscheint.
Wer gerne einen konzeptionelleren Beweis lesen möchte, kann dies entweder in
\cite[Kap.~7]{lecnotes:gub:alg2} oder in \cite[Kap.~22]{lecnotes:jan:la2} tun.
In der letzten Referenz wird insbesondere auf den Seiten~102--121 alles Nötige
entwickelt, um den dann folgenden Beweis auch ohne von der der Linearen Algebra~II
abweichendes Vorwissen verstehen zu können.

\bigskip
In diesem Skript wird folgende Notation verwendet:
\begin{itemize}
    \item
        Sowohl $\subset$ als auch $\subseteq$ stehen für: enthalten oder gleich.
        Echt enthalten wird durch $\subsetneq$ gekennzeichnet.

    \item
        Die \emph{Natürlichen Zahlen $\N$} beginnen mit $0$.

    \item
        Äquivalenzklassen werden durch Überstriche gekennzeichnet, zum Beispiel:
        $\bar 1, \bar c \in \ZRest3$.

    \item
        Zu einem Ring $R$ bezeichnet $R^\times$ die Einheitengruppe des Rings.
\end{itemize}
