
\chapter{Der Elementarteilersatz}
\begin{thSatz}[Elementarteilersatz]
    Sei $F$ ein endlich erzeugter freier Modul über einem HIR $R$. Sei außerdem
    $M\subset F$ ein Untermodul. Dann existieren Elemente $x_1,\ldots,x_s\in F$
    und $\alpha_1,\ldots,\alpha_s\in R\setminus\{0\}$ mit folgenden
    Eigenschaften:
    \begin{enumerate}[i)]
        \item
            $x_1,\ldots,x_s$ sind Teil einer Basis von $F$.
        \item
            $\alpha_1 x_1, \ldots, \alpha_s x_s$ bilden eine Basis von $M$.
        \item
            Es gilt für alle $i\in\{1,\ldots,s-1\}$:\; $\alpha_i\mid\alpha_{i+1}$.
    \end{enumerate}
    Außerdem sind die $\alpha_1,\ldots,\alpha_s$ bis auf Assoziiertheit
    eindeutig durch $M$ bestimmt (unabhängig von der Wahl der $x_1,\ldots,x_s$)
    und man nennt diese Elemente dann die \emph{Elementarteiler} von 
    $M\subset F$.
\end{thSatz}

Der Beweis dieses Satzes wird für die Existenz- und Eindeutigkeitsaussage
getrennt geführt und erfordert einigen Aufwand. Es soll hier deshalb nur eine
Beweisskizze gegeben werden, in der die wichtigen Schritte dargestellt werden.
Einen kompletten Beweis nach dem folgenden Schema kann man in 
% TODO: \cite
[TODO]
nachlesen.

\begin{proofsketch}[ zur Existenzaussage]
    Wir wählen zunächst eine Basis $Y = (y_1,\ldots,y_m)$ von~$F$. Per Induktion
    über $m$ zeigt man nun, dass auch jeder Untermodul $M\subset F$ endlich
    erzeugt ist: für $m=0$ ist nichts zu zeigen und für $m=1$ ist die Behauptung
    klar, da dann $F\cong R$ ist und somit gilt, dass $M$ gerade einem Ideal in
    $R$ entspricht, welches aber nach Voraussetzung von einem einzigen Element erzeugt
    wird; Für $m>1$ betrachtet man dann eine geeignete Zerlegung $F = F'
    \oplus F''$, wendet auf $M\cap F'$ und die Projektion von $M$ auf $F''$ die
    Induktionsvoraussetzung an und zeigt dann, dass man als Vereinigung von
    endlichen Erzeugendensystemen dieser beiden Untermoduln ein
    Erzeugendensystem für ganz $M$ erhält.
    
    Da wir nun wissen, dass $M$ endlich erzeugt ist, können wir ein solches
    Erzeugendensystem $z_1,\ldots,z_n$ wählen. Weiter bezeichne $e_1,\ldots,e_n$
    die Standardbasis von $R^n$ als $R$-Modul, so kann man eine lineare
    Abbildung $f$ wie folgt definieren:
    \begin{align*}
        f\colon R^n &\to F  \\
        e_j &\mapsto z_j
    \end{align*}
    <++>
    \\
\end{proofsketch}












